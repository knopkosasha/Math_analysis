\documentclass[a4paper,12pt,fleqn]{article}

%% Работа с русским языком

		% русские буквы в формулах
\usepackage[T2A]{fontenc}			% кодировка
\usepackage[utf8]{inputenc}			% кодировка исходного текста
\usepackage[english,russian]{babel}	% локализация и переносы

%% Отступы между абзацами и в начале абзаца 
\setlength{\parindent}{0pt}
\setlength{\parskip}{\medskipamount}

%% Изменяем размер полей
\usepackage[top=0.5in, bottom=0.75in, left=0.625in, right=0.625in]{geometry}

%% Графика
\usepackage{graphicx}
\usepackage{wrapfig}

%% Различные пакеты для работы с математикой
\usepackage{amssymb}				% Математические символы
\usepackage{amsthm}					% Пакет для написания теорем
\usepackage{amstext}
\usepackage{array}
\usepackage{amsfonts}
\usepackage{icomma}					% "Умная" запятая: $0,2$ --- число, $0, 2$ --- перечисление
\usepackage{bbm}				    % Для красивого (!) \mathbb с  буквами и цифрами
\usepackage{enumitem}               % Для выравнивания itemise (\begin{itemize}[align=left])
\usepackage{amsmath}


% Ссылки
\usepackage[colorlinks=true, urlcolor=blue]{hyperref}

% Шрифты
\usepackage{euscript}	 % Шрифт Евклид
\usepackage{mathrsfs}	 % Красивый матшрифт

% Перенос знаков в формулах (по Львовскому)
\newcommand*{\hm}[1]{#1\nobreak\discretionary{}
{\hbox{$\mathsurround=0pt #1$}}{}}

% Графики
\usepackage{tikz}
\usepackage{pgfplots}
%\pgfplotsset{compat=1.12}

% Изменим формат \section и \subsection:
\usepackage{titlesec}
\titleformat{\section}
{\vspace{0cm}\centering\LARGE\bfseries}	% Стиль заголовка
{}										% префикс
{0pt}									% Расстояние между префиксом и заголовком
{} 										% Как отображается префикс
\titleformat{\subsection}				% Аналогично для \subsection
{\Large\bfseries}
{}
{0pt}
{}

% Информация об авторах
\author{ПММ ФИИТ 2016-2020
		\\ Маслакова Александра}
\title{Лекции по предмету \\
	\textbf{Математический анализ}}
\date{2016-2017 гг}

\newtheorem*{Def}{Определение}
\newtheorem{Lemma}{Лемма}
\newtheorem*{Consequence}{Следствие}
\newtheorem*{Theorem}{Теорема}
\newtheorem*{Properties}{Свойства}
\newtheorem*{Example}{Пример}
\theoremstyle{remark}
\newtheorem*{Comment}{Замечание}

\newcommand{\D}{\,\mathrm{d}}

\begin{document}
	\section*{Первообразная функции}
	Простейшей, но весьма важной задачей является вопрос об отыскании функции
	$F$ по известной её производной. Пусть $\Delta$ --- конечный или бесконечный
	промежуток числовой оси, на котором заданы $f$ и $F$.
	\begin{Def}
	Функция $F$ называется первообразной для функции $f$ на промежутке $\Delta$,
	если $F$ дифференцируема на промежутке $\Delta$ и в каждой точке этого промежутка
	$F'=f(x)$ (1).
	\end{Def}
	Например, функция $F(x)=\frac{x^3}{3}$ является первообразной для функции $f(x)=x^2$.
	Действительно, $\left(\frac{x^3}{3}\right)'=\frac{3x^2}{3}=x^2$. 
	Однако, $G(x)=\frac{x^3}{3}+1$ также будет первообразной для $f(x)=x^2$. 
	В самом деле, $\left(\frac{x^3}{3}+1\right)'=x^2$. 
	
	Первообразная любой функции неперывна, так как она имеет производную. 
	Функция же, у которой существует первообразная, не обязательно непрерывна.
	\begin{Lemma}
		Для того, чтобы две дифференцируемые на некотором промежутке функции были
		первообразными одной и той же функции, необходимо и достаточно, чтобы они
		на этом промежутке отличались на константу.
	\end{Lemma}
\begin{proof}[Доказательство:]
	Функции $F(x), \Phi(x)$ являются первообразными на промежутке $\Delta$ одной
	и той же функции тогда и только тогда, когда $\Phi(x)=F(x)+C$ при $x\in \Delta$.
	
	Достаточность: если $F$ --- первообразная функции $f(x)$, то $F'(x)=f(x)$.
	$F(x)+C$ также является первообразной для $f(x)$, так как $(F+C)'=F'+C'=F'=f(x)$.
	
	Необходимость: если $F$ и $\Phi$ --- первообразные, то должно выполняться
	равенство $F'=\Phi'=f$. Тогда $(F-\Phi)'=F'-\Phi'=0$, а следовательно, 
	согласно следствию	из теоремы Лагранжа, разность $F-\Phi=C$, C=const.
\end{proof}
\begin{Def}
	Пусть $f$ задана на некотором промежутке $\Delta$.
	Совокупность всех её первообразных на этом промежутке называется 
	неопределённым	интегралом и обозначается $\int f(x)\,\mathrm{d}x$ (2).
\end{Def}
Здесь $f(x)$ --- подынтегральная функция, $f(x)\,\mathrm{d}x$ --- 
подынтегральное выражение, $x$ --- переменная интегрирования.

Из определения неопределённого интеграла следует, что $F(x)$ --- какая-либо 
первообразная для функции $f(x)$, а $\int f(x)\,\mathrm{d}x = F(x)+C$ (3). 
Таким образом, неопределённый интеграл от функции $f$ представляет собой 
общий вид функции с производной $f$.
Кроме того, под знаком интеграла стоит дифференциал функции $F$: $\mathrm{d}F(x)=F'(x)\,\mathrm{d}x=f(x)\,\mathrm{d}x$.
Следовательно, будем считать (по определению дифференциала), что этот
дифференциал под знаком интеграла можно записывать в любом из следующих видов:
$\int f(x)\,\mathrm{d}x=\int F'(x)\mathrm{d}x=\int\mathrm{d}F(x)$ (4).
\subsubsection*{Основные свойства интегралов:}
Будем полагать, что все рассматриваемые функции определены на промежутке $\Delta$.
\begin{enumerate}[start=1]
	\item Если $F$ дифференцируема на $\Delta$, то $$\int\mathrm{d}F(x)=F(x)+C\eqno(5)$$
	\item Пусть $f$ имеет первообразную на $\Delta$. Тогда $$\mathrm{d}\int f(x)\,
	\mathrm{d}x = f(x)\,\mathrm{d}x\eqno(6)$$
\end{enumerate}
Формулы (5) и (6) устанавливают взаимность операций дифференцирования и
неопределённого интегрирования. 
Эти действия взаимно обратны с точностью до константы.
\begin{enumerate}[resume]
	\item Если $f_1,f_2$ имеют первообразную на промежутке $\Delta$, то 
	$f_1+f_2$ тоже имеет первообразную на $\Delta$. $$\int(f_1+f_2)\,\mathrm{d}x=
	\int f_1(x)\,\mathrm{d}x+\int f_2(x)\,\mathrm{d}x$$.
\end{enumerate}
Пусть $F_1, F_2$ --- первообразные функций $f_1$ и $f_2$ соответственно.
Тогда на промежутке $\Delta$ будут справедливы равенства: $F_1'(x)=f_1(x),
F_2'(x)=f_2(x)$.
Тогда неопределённые интегралы $\int f_1(x)\,\mathrm{d}x, \int f_2(x)\,\mathrm{d}x$
будут состоять из функций вида $F_1(x)+C$ и $F_2(x)+C$.

Пусть $F(x)=F_1(x)+F_2(x)$. Тогда $F(x)$ --- первообразная функции $(f_1+f_2)$,
так как $F'(x)=$ $=(F_1+F_2)'=F_1'+F_2'=f_1+f_2$, а интеграл $\int(f_1+f_2)\,
\mathrm{d}x=F(x)+C=F_1(x)+F_2(x)+C$, в то время как 
$\int f_1\,\mathrm{d}x +\int f_2\,\mathrm{d}x=F_1(x)+C_1+F_2(x)+C_2$. 
В силу того, что $C,\,C_1,\,C_2$ --- произвольные постоянные,
то оба множества совпадают.
\begin{enumerate}[resume]
	\item Если $f$ имеет первообразную и $k$ --- некоторое число, то $k\cdot f$
	также имеет первообразную при $k\neq0$, тогда $\int (k\cdot f(x))\,\mathrm{d}x
	=k\cdot \int f(x)\,\mathrm{d}x$.
\end{enumerate}
Пусть $F$ --- первообразная функции $f(x)$. Тогда $k\cdot F(x)$ --- первообразная
для $k\cdot f(x)$, поскольку $(k\cdot F(x))'=k\cdot F'(x)=k\cdot f(x)$. 
Поэтому $\int (k\cdot f(x))\,\mathrm{d}x=k\cdot F(x)+C=k\cdot\int f(x)\,
\mathrm{d}x=k\cdot(F(x)+C)=$ $=k\cdot F(x)+k\cdot C$. 

\begin{Consequence}[линейность]
	Если $f_1,f_2$ имеют первообразную на $\Delta$, а $\lambda_1\in\mathbb{R},
	\lambda_2\in\mathbb{R}$, то 
	$$\int(\lambda_1f_1+\lambda_2f_2)\,\mathrm{d}x=\lambda_1\int f_1(x)\mathrm{d}x
	+\lambda_2\int f_2(x)\mathrm{d}x$$
\end{Consequence}
\subsection*{Интеграл и задача об определении площади}
\begin{wrapfigure}{l}{250pt}
	\includegraphics[width=0.5\textwidth]{integral}
\end{wrapfigure}
Пусть дана $y=f(x)$ на отрезке $[a;b]$, принимающая лишь положительные значения.
Рассмотрим фигуру $ABCD$, которая ограничена кривой $DC$, двумя ординатами $x=a$
и $x=b$ и отрезком оси $Ox$. Такую фигуру будем называть криволинейной трапецией.

Желая определить величину площади этой фигуры, изучим поведение площади переменной
фигуры $AKLD$, которая заключена между ординатой $a$ и ординатой, соответствующей
произвольной $x$ из отрезка $[a;b]$. При изменении $x$ площадь этой фигуры будет 
изменяться соответствующе. Следовательно, площадь трапеции $AKLD$ --- некоторая
функция, зависящая от $x$. $S_{AKLD}=S(x)$.

Сначала найдём производную этой функции.
Придадим $x$ некоторое приращение $\Delta x$, тогда площадь получит приращение $\Delta S$.
Обозначим через $m$ и $M$ наименьшее и наибольшее значение $f(x)$ в промежутке 
$[x; x+\Delta x]$. Сравним площадь $\Delta S$ с площадью прямоугольников,
построенных на основании $\Delta x$ и имеющих высоты $m$ и $M$. 
$m\cdot\Delta x < \Delta S < M\cdot\Delta x$. Разделим всё на $\Delta x$:
$m<\frac{\Delta S}{\Delta x}<M$. Если $\Delta x\to 0$, то, в силу непрерывности
функции, $m$ и $M$ будут стремиться к $f(x)$, тогда 
$S'(x) = \lim_{\Delta x\to 0}\frac{\Delta S}{\Delta x} = f(x)$.
Таким образом, приходим к  теореме Ньютона-Лейбница:
\begin{Theorem}[Ньютона-Лейбница]
	Производная от переменной площади $\Delta S$ по конечной абсциссе равна конечной
	ординате $y=f(x)$.
\end{Theorem}
Другими словами, переменная площадь $S(x)$ представляет собой первообразную
фнукцию для заданной функции $y=f(x)$. 

Если известна какая-либо первообразная $F(x)$ для функции $f(x)$, то площадь
$S(x)$ равна $F(x)+C$. Постоянную $C$ легко определить, положив $x=a$. 
$F(a)+C=0$, следовательно, $C = -F(a)$.  $S(x)=F(x)-F(a)$, и, в частности,
для получения площади трапеции $ABCD$ необходимо принять $x=b$. Получим:
$$S_{ABCD}=F(b)-F(a)$$

\end{document}