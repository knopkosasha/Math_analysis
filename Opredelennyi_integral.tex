\documentclass[a4paper,12pt,fleqn]{article}

%% Работа с русским языком

		% русские буквы в формулах
\usepackage[T2A]{fontenc}			% кодировка
\usepackage[utf8]{inputenc}			% кодировка исходного текста
\usepackage[english,russian]{babel}	% локализация и переносы

%% Отступы между абзацами и в начале абзаца 
\setlength{\parindent}{0pt}
\setlength{\parskip}{\medskipamount}

%% Изменяем размер полей
\usepackage[top=0.5in, bottom=0.75in, left=0.625in, right=0.625in]{geometry}

%% Графика
\usepackage{graphicx}
\usepackage{wrapfig}

%% Различные пакеты для работы с математикой
\usepackage{amssymb}				% Математические символы
\usepackage{amsthm}					% Пакет для написания теорем
\usepackage{amstext}
\usepackage{array}
\usepackage{amsfonts}
\usepackage{icomma}					% "Умная" запятая: $0,2$ --- число, $0, 2$ --- перечисление
\usepackage{bbm}				    % Для красивого (!) \mathbb с  буквами и цифрами
\usepackage{enumitem}               % Для выравнивания itemise (\begin{itemize}[align=left])
\usepackage{amsmath}


% Ссылки
\usepackage[colorlinks=true, urlcolor=blue]{hyperref}

% Шрифты
\usepackage{euscript}	 % Шрифт Евклид
\usepackage{mathrsfs}	 % Красивый матшрифт

% Перенос знаков в формулах (по Львовскому)
\newcommand*{\hm}[1]{#1\nobreak\discretionary{}
{\hbox{$\mathsurround=0pt #1$}}{}}

% Графики
\usepackage{tikz}
\usepackage{pgfplots}
%\pgfplotsset{compat=1.12}

% Изменим формат \section и \subsection:
\usepackage{titlesec}
\titleformat{\section}
{\vspace{0cm}\centering\LARGE\bfseries}	% Стиль заголовка
{}										% префикс
{0pt}									% Расстояние между префиксом и заголовком
{} 										% Как отображается префикс
\titleformat{\subsection}				% Аналогично для \subsection
{\Large\bfseries}
{}
{0pt}
{}

% Информация об авторах
\author{ПММ ФИИТ 2016-2020
		\\ Маслакова Александра}
\title{Лекции по предмету \\
	\textbf{Математический анализ}}
\date{2016-2017 гг}

\newtheorem*{Def}{Определение}
\newtheorem{Lemma}{Лемма}
\newtheorem*{Consequence}{Следствие}
\newtheorem*{Theorem}{Теорема}
\newtheorem*{Properties}{Свойства}
\newtheorem*{Example}{Пример}
\theoremstyle{remark}
\newtheorem*{Comment}{Замечание}

\newcommand{\D}{\,\mathrm{d}}

\begin{document}
	\section{Определённый интеграл}
	Рассмотрим задачу о движении точки вдоль числовой оси.
	Пусть $S(t)$ --- её координата в момент времени $t$, а
	$v(t) = S'(t)$ --- её скорость в тот же момент времени.
	Предположим, что мы знаем $S(t_0)$ точки в момент времени $t_0$, и пусть нам
	поступают данные о её скорости, и мы хотим вычислить $S(t)$ для любого
	фиксированного времени $t>t_0$.
	Если считать скорость $v(t)$ меняющейся непрерывно, то смещение точки за
	малый промежуток времени $\Delta t$ можно вычислить как произведение
	$v(\tau)\cdot\Delta t$, где $\tau$ --- произвольный момент времени.
	
	Разобъём отрезок $[t_0;t]$, отметив некоторые моменты времени $t_i$, такие,
	что	$t_0<t_1<\ldots<t_i<\ldots<t_n$, причём промежутки $[t_{i-1};t_i]$ малы,
	а $\tau\in[t_{i-1};t_i]$. Тогда будем иметь приближенное равенство:
	\[
	S(t)-S(t_0)\approx\sum^{n}_{i=1}v(t_i)\Delta t
	\]
	Это приближенное равенство будет уточняться, если переходить к разбиениям
	отрезка $[t_0;t]$ на всё более мелкие промежутки.
	Таким образом,в пределе, когда величина наибольшего из промежутков разбиения
	будет стремиться к нулю, \[\lim\limits_{d\to 0}\sum_{i=1}^{n}v(\tau_i)\Delta t
	=S(t)-S(t_0)\]
	Сумма, стоящая в левой части равенства, называется интегральной суммой.
	Отметим, что это равенство есть не что иное, как фундаментальная формула для
	матанализа, называемая формулой Ньютона-Лейбница.
	Она позволяет, в частности, находить первообразную $S(t)$ по её производной
	$v(t)$.
	\subsection{Понятие интегральной суммы и её предела}
	Пусть функция $f(x)$ определена и ограничена на отрезке $[a;b]$.
	Рассмотрим конечное число точек $x_1\ldots x_{n-1}$, лежащих внутри отрезка,
	удовлетворяющих неравенству $a<x_1<\ldots <x_{n-1}<b$. Положим $a = x_0,
	b = x_n$. Тогда указанные точки производят разбиение отрезка $[a;b]$ на $n$
	частичных отрезков $[x_0;x_1], [x_1;x_2],\ldots [x_{n-1};x_n]$.
	Длину $k$-го отрезка обозначим за $\Delta x_k = x_n - x_{n-1}$, возьмём на
	каждом $k$-м отрезке произвольную точку $\xi_k$, такую, что $x_{k-1}\leq\xi_k
	\leq x_k$ и составим  для рассмотренного разбиения следующую сумму:
	\[
	\sigma = \sigma(x_k,\xi_k) = \sum_{i=1}^{n}f(\xi_k)\cdot\Delta x_k
	\]
	Эта сумма называется интегральной суммой для функции $f(x)$ на отрезке $[a;b]$.
	%%РИСУНОК ИНТЕГРАЛА!!!!!!!
	Геометрический смысл $\sigma$ очевиден: это сумма площадей с основаниями 
	$\Delta x_1,\ldots,\Delta x_n$ и высотами $\xi_1,\ldots,\xi_n$, то есть,
	площадь криволинейной трапеции.
	\begin{Def}
		Число $I$ называется пределом интегральных суммм при стремлении к 0
		наибольшей длины $d$ частичных отрезков, если для произвольного
		$\varepsilon>0$ найдётся соответствующее ему $\delta(\varepsilon)$, такое, что
		при единственном условии $d<\delta(\varepsilon)$ справедливо неравенство
		$\sigma - I < \varepsilon$.
		
		$\forall\; x>0\; \exists\;\delta(\varepsilon)\colon d<\delta(\varepsilon)
		\Rightarrow |\sigma-I|<\varepsilon$
	\end{Def}
	
	
	
	
	
	
	
	
	
	
	
	
	
	
	
	
\end{document}